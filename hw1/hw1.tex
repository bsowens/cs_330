
\documentclass[11pt]{article}
\usepackage{graphicx}
\usepackage{fullpage}
\usepackage{hyperref}

\usepackage{times,graphicx,epstopdf,fancyhdr,amsfonts,amsthm,amsmath,algorithm,algorithmic,xspace,hyperref}
\usepackage[left=.75in,top=.75in,right=.75in,bottom=.75in]{geometry}

%\textwidth 7in
%\textheight 9.5in

\pdfpagewidth 8.5in
\pdfpageheight 11in 

\pagestyle{fancy}


% Margins are default 1in. on all sides
% thus, if you wanted .5 on the left, and right
% you'd do \oddsidemargin -.25in.
% \evensidemargin -.25in. and
% \textwidth 7in.
\oddsidemargin 0in
\evensidemargin 0in


\newtheorem{claim}{Claim}
\newtheorem{definition}{Definition}
\newtheorem{theorem}{Theorem}
\newtheorem{lemma}{Lemma}
\newtheorem{observation}{Observation}
\newtheorem{question}{Question}


\begin{document}
	
\begin{center}
\Large{\textbf{CS 330 -- Spring 2016, Assignment 1}}\\
\large{\textbf{Problems due 7PM, Thursday January 28.  Please submit a hardcopy in the CS drop box.}}
\end{center}

\lfoot{Boston University}
\rfoot{Prof. John Byers}

Our first homework has several questions that require short proofs.  In writing up proofs, try to make 
sure your reasoning flows logically from one statement to another.  You should edit your proofs to 
make sure they are clear and concise.  Typesetting your solutions in \LaTeX (www.latex-project.org) is
preferred, but you may also write them up. 


\setcounter{question}{-1}


\begin{question}
Chapter 2, Exercise 4, on p. 67.  (Review question - not graded, do not submit).
\end{question}



\begin{question}
Exercises 1 and 2 of Chapter 1 on p. 22.
\end{question}

\begin{quotation}

1.
"True or false? In every instance of the Stable Matching Problem, there is a stable matching containing a pair (m, w) such that m is ranked first on the preference list of w and w is ranked first on the preference list of m."

\end{quotation}

\begin{large}
False.
Counterexample: the first example from the lecture slides

\begin{table}[h]
\begin{center}
\begin{tabular}{r|ccc}
(m, & \multicolumn{3}{c}{w)}\\
\hline
X & A & B & C \\
Y & B & A & C \\
Z & A & B & C
\end{tabular}
\end{center}
\end{table}
\begin{table}[h]
\begin{center}
\begin{tabular}{r|ccc}
(w, & \multicolumn{3}{c}{m)}\\
\hline
A & Y & X & Z \\
B & X & Y & Z \\
C & X & Y & Z
\end{tabular}
\end{center}
\end{table}

\end{large}


\begin{quotation}
2.
True or false? Consider an instance of the Stable Matching Problem in which there exists a man m and a woman w such that m is ranked first on the preference list of w and w is ranked first on the preference list of m. Then in every stable matching S for this instance, the pair (m, w) belongs to S.
\end{quotation}

\begin{large}
True, because, in the Gale-Shapley algorithm, this "perfect pair" will be either matched immediately, or upon finding that the male is matched with a lower-order female, his match will be switched to the \#1 female.
\end{large}
\pagebreak 

\begin{question}
Consider the following problem called the {\em Stable Officemate Problem}.  Here there are $2n$ people, each of whom ranks the other $2n-1$ people in order of preference.  For example, Addie, Brian, Carlos, and Don might have the following rankings:

\begin{table}[h]
\begin{center}
\begin{tabular}{r|ccc}
Name & \multicolumn{3}{c}{Preference}\\
\hline
Addie & Brian & Carlos & Don\\
Brian & Addie & Carlos & Don \\
Carlos & Don & Addie & Brian \\
Don & Carlos & Addie & Brian
\end{tabular}
\end{center}
\end{table}

The goal is to find a stable matching ({\em i.e.,} $n$ pairs of officemates such that no two people prefer each other to their current officemate).  We proved in class that a stable matching always exists for instances of the stable matching problem.  Do stable matchings always exist for the stable officemate problem?  If so, provide a proof.  If not, give a counterexample.
\end{question}
\pagebreak 
\begin{question}
Recall the {\em largest sum subinterval} problem.  The input is an array $A$ of integer values.  The output is the largest sum of any subinterval of $A$.  In class we will discuss both a cubic ($O(n^{3})$) and a quadratic ($O(n^{2})$) time algorithm for solving this problem.  Here you will develop an $O(n)$ time algorithm to solve it.
\begin{enumerate}
	\item Let $PS(j) = \sum_{i=1}^{j} A[i]$ give the partial sum of the first $j$ integer values of $A$.  Show that if $PS(j) \geq 0$ for all $1 \leq j \leq n$, then the largest sum subinterval is the interval $[1,k]$ where $k$ maximizes $PS(k)$.
	\item Show that when $PS(j)$ falls below 0, the problem essentially ``resets''  with $PS(j)$ being the new 0.  That is, the largest sum subinterval never includes $j$---it falls on one side or the other.
	\item Use the above two claims to give an elegant $O(n)$-time solution to the largest sum subinterval problem.
\end{enumerate}
\end{question}

\begin{question}
Chapter 2, Exercise 8, on pp. 69-70.  For part (b), we will just grade the $k=3$ case, but feel free to solve it in general.
\end{question}


\end{document}
